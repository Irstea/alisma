\chapter{Introduction}

\section{Objectifs du logiciel}
Le logiciel ALISMA est conçu pour permettre la saisie des relevés des macrophytes réalisés en cours d'eau, dans le cadre de la DCE \og Cours d'eau \fg{}. 

Il permet le calcul de l'indicateur.
\subsection{Technologie employée}
Le logiciel a été écrit en Java. Il fonctionne avec une base de données MySql, qui peut être hébergée localement ou en réseau (les paramètres sont facilement modifiables).
\subsection{Clauses de propriété}
Le logiciel est distribué sous la licence GNU GPL v3 (\url{http://www.gnu.org/licenses/gpl.txt}).

La version 1.0 a été déposée à l'Agence de Protection des Programmes (\url{https://www.app.asso.fr}), sous le numéro \textbf{IDDN.FR.001.270035.000.S.C.2016.000.31500}

\section{Assistance et maintenance}

\subsection{Limite de responsabilité - sécurité}

Le logiciel est fourni \textit{en l'état}, il ne peut être fait grief à Irstea d'un quelconque problème, d'une perte ou d'une corruption des données, ou d'un dysfonctionnement.

Il appartient à l'organisme utilisateur d'assurer la sécurité de ses données tant en confidentialité qu'en intégrité ou en disponibilité, selon les besoins de sécurité qu'il aura déterminé. 

En particulier, même si l'application propose une fonction de sauvegarde de la base de données qui est activée régulièrement, il appartient à l'utilisateur de vérifier qu'elle s'exécute correctement. Il devra également la copier vers un support protégé pour éviter tout risque de perte accidentelle.

\subsection{Dysfonctionnement applicatif}

En cas de dysfonctionnement de l'application, il est possible d'ouvrir un ticket dans \textit{github}, à l'adresse \url{https://github.com/Irstea/alisma/issues/new}.

Irstea ne prend pas d'engagement concernant leur traitement ou leur prise en compte.
