\chapter{Introduction}

\section{Clause de responsabilité}

Irstea décline toute responsabilité quant au contenu des informations présentes dans ce document, et notamment de toute interprétation qui pourrait en être faite. Il appartient aux utilisateurs de s'assurer que les opérations ou les modes opératoires décrits sont bien cohérents avec les actions qu'ils souhaitent mener. 

Irstea ne pourra être tenu pour responsable de tout dommage, perte, etc. qui pourrait être imputable à la lecture de ce document.

\section{Objectifs du logiciel}
Le logiciel ALISMA est conçu pour permettre la saisie des relevés des macrophytes réalisés en cours d'eau, dans le cadre de la DCE \og Cours d'eau \fg{}. 

Il permet le calcul de l'indicateur.

Pour toutes informations concernant le protocole, consultez le site \href{https://hydrobio-dce.irstea.fr/cours-deau/macrophytes/}{hydrobio-dce.irstea.fr/cours-deau/macrophytes/}

\subsection{Technologie employée}
Le logiciel a été écrit en Java. Il fonctionne soit en mode \textit{monoposte} avec une base de données embarquée HSQLDB (fonctionnement transparent pour l'utilisateur), soit en mode réseau, avec une base de données MySQL, pour un partage des informations entre plusieurs utilisateurs.

\subsection{Clauses de propriété}
Le logiciel est distribué sous la licence GNU GPL v3 (\url{http://www.gnu.org/licenses/gpl.txt}).

La version 1.0 a été déposée à l'Agence de Protection des Programmes (\url{https://www.app.asso.fr}), sous le numéro \textbf{IDDN.FR.001.270035.000.S.C.2016.000.31500}

\section{Assistance et maintenance}

\subsection{Limite de responsabilité - sécurité}

Le logiciel est fourni \textit{en l'état}, il ne peut être fait grief à Irstea d'un quelconque problème, d'une perte ou d'une corruption des données, ou d'un dysfonctionnement.

Il appartient à l'organisme utilisateur d'assurer la sécurité de ses données tant en confidentialité qu'en intégrité ou en disponibilité, selon les besoins de sécurité qu'il aura déterminé. 

Selon le mode de fonctionnement, il devra soit réaliser une copie du dossier contenant la base de données (mode autonome -- HSQLDB) soit s'assurer que les sauvegardes sont correctement réalisées et recopier les fichiers générés sur un support tiers (mode réseau -- MySQL).

En particulier, même si l'application propose une fonction de sauvegarde de la base de données qui est activée régulièrement, il appartient à l'utilisateur de vérifier qu'elle s'exécute correctement. Il devra également la copier vers un support protégé pour éviter tout risque de perte accidentelle.

\subsection{Dysfonctionnement du logiciel}

En cas de dysfonctionnement de l'application, il est possible d'ouvrir un ticket dans \textit{github}, à l'adresse \url{https://github.com/Irstea/alisma/issues/new}.

Irstea ne prend pas d'engagement concernant leur traitement ou leur prise en compte.

\section{Description des versions}
Le détail des versions est décrit dans le fichier \textit{news.txt}, présent à la racine de l'application.
