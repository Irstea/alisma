\chapter{Installer le logiciel}

\section{Installer les pré-requis}
\subsection{JRE Java}
Si nécessaire (votre ordinateur dispose peut-être déjà d'une version du JRE), téléchargez le programme d'installation du JRE Java à partir du site : \url{http://www.oracle.com/technetwork/java/javase/downloads/index.html}

\subsection{Serveur MySql -- mode réseau}

Si la base de données a vocation à être partagée entre plusieurs utilisateurs, le fonctionnement avec un serveur MySQL est possible. Pour tous renseignements concernant l'installation et le fonctionnement de MySQL, consultez le site \url{https://www.mysql.com/}.

Vous devriez également installer SQL Workbench, le client fourni par MySQL pour manipuler les bases de données et les comptes des utilisateurs.

Dans le dossier d'installation de MySql, ouvrez le fichier \textit{my.cnf} ou \textit{my.ini} (en principe, dans le dossier : C:\textbackslash{} ProgramData\textbackslash{}MySQL\textbackslash{}MySQL Server 5.7 pour les installations sur des plates-formes Windows. Dans la section [mysqld], recherchez l'entrée \textit{max\_allowed\_packet}, et vérifiez que la valeur soit au minimum à 16M :
\begin{lstlisting}
[mysqld]
...
max_allowed_packet      = 16M
\end{lstlisting}


Relancez ensuite le serveur Mysql.

Connectez-vous au serveur MySQL, et créez :
\begin{itemize}
\item un utilisateur dédié à la base de données, par exemple \textit{alisma}, avec le mot de passe \textit{alisma} ;
\item la base de données \textit{alisma}.
\end{itemize}

Assurez-vous que l'utilisateur \textit{alisma} dispose bien de l'ensemble des droits pour la base de données \textit{alisma}.

\subsection{Client de bases de données}

Si vous avez besoin de réaliser des requêtes d'interrogation dans la base de données, vous pouvez installer le logiciel Java SQL Workbench/J (\url{http://www.sql-workbench.net/}), qui vous permettra de vous connecter aussi bien à la base de données HSQLDB (fonctionnement monoposte) que MySQL (fonctionnement en réseau). Pour la configuration des connexions dans SQL Workbench/J, consultez la procédure décrite en annexe, page \pageref{sqlworkbenchj}.


\section{Installer le logiciel}

\subsection{Précautions à prendre en cas de mise à jour}
Si vous réalisez une mise à jour du logiciel, pensez à sauvegarder :
\begin{itemize}
\item le dossier \textit{\NoAutoSpaceBeforeFDP c:\textbackslash{}alisma\textbackslash{}data}, qui contient la base de données embarquée (fonctionnement monoposte) ;
\item le fichier \textit{\NoAutoSpaceBeforeFDP c:\textbackslash{}alisma\textbackslash{}param\textbackslash{}param.ini}, qui contient vos paramètres personnels.
\end{itemize}
Notez également la version actuelle de l'application, que vous retrouverez dans la barre d'état de la fenêtre du logiciel.

\subsection{Décompresser l'archive}

Décompressez l'archive à la racine du disque c:, qui comprend un dossier \textit{alisma}.

en cas de mise à jour, remplacez le dossier \textit{data} par celui préalablement sauvegardé.

\section{Configurer la base de données}

\subsection{Créer la base de données}

Cette opération doit être réalisée pour toute nouvelle installation, et uniquement dans ce cas.

\subsubsection{Base de données embarquée}

Dans le dossier \textit{\NoAutoSpaceBeforeFDP c:\textbackslash{}alisma\textbackslash{}hsqldb}, exécutez le programme \textit{create.bat}. Il va créer la structure de la base de données.

\subsubsection{Base de données MySQL}

Dans le dossier \textit{\NoAutoSpaceBeforeFDP c:\textbackslash{}alisma\textbackslash{}mysql}, éditez le fichier \textit{create.bat}. Vérifiez notamment les paramètres de connexion (nom du serveur, compte, mot de passe, nom de la base de données, etc.) et l'emplacement du client MySQL dans la machine.

Une fois les modifications effectuées, exécutez le script, qui créera la structure de la base de données.

\section{Mettre à jour la base de données}

Cette opération ne doit être réalisé que si le logiciel a déjà été installé dans votre poste de travail (mode monoposte), ou si la base de données est hébergée dans un serveur (mode réseau).

Les scripts de mise à jour sont nommé ainsi : \textit{alter-xx-yy.bat}, où :
\begin{itemize}
\item \textit{xx} correspond à la version à mettre à jour ;
\item \textit{yy} correspond à la version cible.
\end{itemize}

Vous devrez exécuter tous les scripts, les uns après les autres, en recherchant celui dont la valeur \textit{xx} est égale ou strictement inférieure à la version du logiciel précédemment installé.

Par exemple, si vous disposiez du logiciel en version 1.1 et que vous installez la version 1.2, vous devrez exécuter le script \textit{alter-1.1-1.2.bat}.

\subsubsection{Base de données embarquée -- mode monoposte}

Les scripts sont dans le dossier \textit{\NoAutoSpaceBeforeFDP c:\textbackslash{}alisma\textbackslash{}hsqldb}.

\subsubsection{Base de données serveur -- mode réseau}

Les scripts sont dans le dossier \textit{\NoAutoSpaceBeforeFDP c:\textbackslash{}alisma\textbackslash{}mysql}. Avant de lancer un script, assurez-vous que les paramètres de connexion soient corrects, en l'éditant préalablement.

\section{Adapter les paramètres à votre installation}
\label{sec:param}

Si vous fonctionnez en mode monoposte, vous n'aurez probablement besoin de ne  modifier que les rubriques \textit{preleveur, organisme, operateur, producteur et determinateur} de la section \textit{others} (valeurs par défaut pré-positionnées lors de la saisie d'un nouveau relevé).

En cas d'évolution des serveurs fournissant les référentiels ou calculant l'indice officiel, vous pourriez être amenés à modifier les paramètres correspondants. Consultez le cas échéant vos correspondants dans les agences de l'Eau ou à l'Agence française de la bio-diversité pour de plus amples informations.

Le fichier \textit{param\textbackslash{}param.ini} contient les paramètres qui peuvent être modifiés facilement par l'utilisateur. Néanmoins, la plus grande prudence est de mise, le risque de créer des dysfonctionnements ultérieurement n'étant alors pas nul...

Pour modifier le fichier \textit{param.ini}, utilisez \textit{notepad++} (\url{https://notepad-plus-plus.org}) plutôt que \textit{notepad}, pour éviter les problèmes de caractères de fin de ligne, qui sont différents entre Windows et les autres systèmes d'exploitation.

Le fichier de paramètres est organisé en sections. Les valeurs par défaut précisées ici sont celles d'une installation monoposte. Pour une utilisation en mode réseau, remplacez le fichier \textit{param.ini} par une copie du fichier \textit{param-mysql.ini}.

\subsection{database}
Cette section précise les paramètres de connexion à la base de données.
\begin{description}
\item [dbuser] : nom du compte de connexion (\textit{SA})
\item [dbpass] : mot de passe de connexion ()
\item [jdbc\_class] : nom du pilote d'accès à la base de données (\textit{org.hsqldb.jdbc.JDBCDriver})
\item [jdbc\_string] : chaîne de connexion à la base de données (jdbc:hsqldb:file:data/alisma)
\item [dbtype] : type de la base de données (\textit{hsqldb})
\item [dbColumnSeparator] : séparateur interne utilisé pour les noms de colonnes (vide)
\item [dbencode\_iso8859] : true|false. Indique si les caractères doivent être encodés en ISO-8859-15 (true). La valeur \textit{true} ne doit être positionnée que pour les postes de travail fonctionnant avec Windows
\item [backupFileNamePrefix] : préfixe utilisé pour les sauvegardes de la base de données serveur (\textit{alisma\_db})
\item [backupProgram] : nom du programme utilisé pour réaliser la sauvegarde de la base de données serveur (vide)
\item [pathFolderDataSave] : chemin de stockage des sauvegardes (\textit{\NoAutoSpaceBeforeFDP c:\textbackslash{}alisma\textbackslash{}backup})
\item [pathFileDateSave] : chemin d'accès au fichier contenant la date de la dernière sauvegarde réalisée (\textit{\NoAutoSpaceBeforeFDP c:\textbackslash{}alisma\textbackslash{}backup\textbackslash{}alisma\_save.txt})
\item [backupDelay] : délai entre deux sauvegardes (-1). Pour que le logiciel déclenche des sauvegardes automatiques, ce paramètre doit être supérieur à -1. Les sauvegardes ne sont pas possibles en mode monoposte.
\end{description}

Si votre base de données est hébergée dans un serveur, contactez votre administrateur informatique pour connaître les paramètres à renseigner.

\subsection{language}

Cette section permet de définir la langue utilisée par défaut dans le logiciel.

\begin{description}
\item [default] : langue utilisée par défaut (fr\_FR)
\item [preferred] : langue préférée (fr\_FR)
\item[en\_US]: langue utilisée en cas de libellé manquant (fr\_FR)
\item [en\_GB] : idem
\end{description}


\subsection{others}

\begin{description}
\item [modeDebug] : active l'affichage des messages dans le logiciel (false)
\item [lambert] : indique si les coordonnées Lambert sont utilisées (true). Si le logiciel est employé pour des relevés hors France, basculez ce paramètre à \textbf{false}
\item [lambert93Emin] : coordonnée Lambert Est minimale (100000)
\item [lambert93Emax] : coordonnée Lambert Est maximale (1200000)
\item [lambert93Nmin] : coordonnée Lambert Nord minimale (6000000)
\item [lambert93Nmax] : coordonnée Lambert Nord maximale (7100000)
\item [pathFolderExport] : chemin de stockage des fichiers générés (\NoAutoSpaceBeforeFDP c:\textbackslash{}alisma\textbackslash{}export)
\item [exportFileNamePrefix] : préfixe utilisé pour les fichiers exportés (alisma)
\item [xsltfile\_fr] : nom du fichier contenant le filtre utilisé pour générer les fichiers PDF, en français (param/alisma.xsl)
\item [xsltfile\_en] : nom du fichier contenant le filtre utilisé pour générer les fichiers PDF, en anglais (param/alisma.xsl)
\item [preleveur\_code] : code par défaut du préleveur
\item [preleveur\_name] : nom par défaut du préleveur
\item [organisme] : nom par défaut de l'organisme réalisant le prélèvement
\item [operateur] : nom par défaut de l'opérateur (codification interne)
\item [producteur\_code] : code par défaut du producteur
\item [producteur\_name] : nom par défaut du producteur
\item [determinateur\_code] : code par défaut du déterminateur
\item [determinateur\_name] : nom par défaut du déterminateur
\item [sandreStationExportAddress] = adresse de récupération automatique de la liste des stations déclarées dans le Sandre

\end{description}

\subsection{fieldslevel}
Cette section permet de rendre certains champs obligatoires ou non. Quatre niveaux sont disponibles :
\begin{itemize}
\item \textit{vide} : pas de contrôle particulier sur le champ ;
\item \textit{recommanded} : le champ concerné devrait être renseigné ;
\item \textit{necessary} : le champ doit être renseigné pour que le dossier soit validé ;
\item \textit{mandatory} : le champ doit être renseigné pour pouvoir enregistrer le dossier.
\end{itemize}

Dans la pratique, ces niveaux ne sont modifiables que pour les informations suivantes :
\begin{description}
\item [coord\_x et coord\_y] : coordonnées Lambert. Les valeurs doivent être vidées si le logiciel est utilisé hors de France (modification à réaliser en même temps pour le paramètre \textit{others/lambert}) 
\item [wgs84\_x et wgs84\_y] : coordonnées WGS84 (recommanded)
\item [lambert\_x\_aval et lambert\_y\_aval] : coordonnées Lambert du point aval (recommanded)
\item [wgs84\_x\_aval et wgs84\_y\_aval] : coordonnées WGS84 du point aval (recommanded)
\end{description}

\subsection{display}
Cette section contient le code des couleurs employées, ainsi que les adresses des logos.
\begin{description}
\item [colorFooter] : couleur du bas d'écran (150,182,254)
\item [colorBanniere] : couleur de la bannière (150,182,254)
\item [colorCentral] : couleur de la fenêtre (157,212,255)
\item [colorTab] : couleur des onglets (0,20,205)
\item [icone] : nom du fichier contenant l'icône de l'application (ressources/alisma.png)
\item [logo] : nom du fichier contenant le logo d'IRSTEA (ressources/logo.png)
\end{description}

\subsection{seee}
Cette section est utilisée pour l'interrogation et le calcul SEEE de l'indicateur.
\begin{description}
\item[url] : adresse du site web (http://seee.eaufrance.fr)
\item [resourceIbmrCalc] : chemin d'accès au programme de calcul (/api/calcul)
\item [indicator] : code de l'indicateur (IBMR)
\item [version] : version utilisée pour le calcul de l'indicateur (1.1.1)
\item [proxyEnabled] : si à \textit{true}, la connexion est réalisée en utilisant un proxy d'entreprise (false)
\item [proxyHost] : nom DNS (ou adresse IP) du serveur Proxy (ne pas préfixer par http://, indiquer uniquement le nom,par exemple \textit{proxy.masociete.com})
\item [proxyPort] : port de connexion au proxy (8080 par défaut)
\item [proxyUser] : si le proxy requiert une identification, code de l'utilisateur concerné
\item [proxyPassword] : mot de passe associé
\end{description}

\section{Sauvegarde des données}

En mode monoposte, aucune sauvegarde n'est réalisée par le logiciel. Vous devrez recopier le dossier \textit{\NoAutoSpaceBeforeFDP c:\textbackslash{}alisma\textbackslash{}data} dans un support protégé, pour vous prémunir soit contre des problèmes techniques, soit contre des erreurs ou des opérations malencontreuses.

En mode réseau, le logiciel peut déclencher une sauvegarde. Pensez également à recopier le fichier généré sur un support protégé pour éviter tout risque de perte.
